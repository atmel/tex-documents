% -*-coding: utf-8 -*-

\section{Vliv masky na výsledek filtrace}\label{diskuse masky}

    Nyní provedeme diskusi zvolené masky na dvě hlavní vlastnosti filtru: odolnost vůči kontaminaci a snižování variance gaussovského šumu. Porovnávat budeme s klasickým průměrovým filtrem, který nejlépe snižuje varianci šumu, ale nemá žádnou odolnost proti kontaminaci a bude nás zajímat, jak se tyto dvě vlastnosti dají efektivně směnit. Nejprve se podíváme na jednodušší filtry medián a BES. U těch má smysl používat binární (viz teorie \ref{statisticky motivované}) masku se sudou kapacitou, neboť pak musíme medián brát jako průměr dvou prostředních prvků, čímž lépe snížíme varianci šumu. Na obrázku~\ref{obr masky} uvádíme nejpoužívanější masky (bez centrálního voxelu) pořadě o kapacitě 6, 18 a 26.

\begin{figure}[h]\label{obr masky}
  \includegraphics[width = \textwidth]{src/5Vysledky/masky.pdf}
  \caption{Nejpoužívanější masky}
\end{figure}

    V tabulce~\ref{tab med BES} uvádíme data pro zmíněné dva filtry. Sloupec \emph{snížení variance šumu} udává, jaká bude variance šumu ve filtrovaném obrázku vůči původní. Pro srovnání uvádíme jak snižuje varianci obyčejný aritmetický průměr nad stejnou maskou.

\begin{table}[h]\label{tab med BES}
    \begin{center}
    \begin{tabular}{lllll}
      \toprule
      \multirow{3}{*}{Filtr$_{\mathrm{\substack{kapacita\\ masky}}}$} & max kontamino- & odolnost    & snížení    & snížení variance \\
                                                    & vaných voxelů   & proti       & variance   & šumu čistým      \\
                                                    & v masce         & kontaminaci & šumu       & průměrováním     \\
      \midrule
      medián$_{\mathrm{6}}$             & 2                 & 33,3 \%       & $1/2$    & $1/6$ \\
      medián$_{\mathrm{18}}$            & 8                 & 44,4 \%       & $1/2$    & $1/18$ \\
      medián$_{\mathrm{26}}$            & 12                & 46,2 \%       & $1/2$    & $1/26$  \\
      medián$_{c}$                      & $\lfloor\frac{c+1}{2}\rfloor-1$& $\le$ 50 \%& 1 až $1/2$& $1/c$\\
      BES$_{\mathrm{6}}$                & 1                 & 16,7 \%       & $1/4$   & $1/6$ \\
      BES$_{\mathrm{18}}$               & 4                 & 22,2 \%       & $1/4$   & $1/18$ \\
      BES$_{\mathrm{26}}$               & 6                 & 23,1 \%       & $1/4$   & $1/26$ \\
      BES$_{c}$                         & $\lceil\frac{c}{4}\rceil-1$& $\le$ 25 \%& $1/3$ až $1/4$& $1/c$\\
      \bottomrule
    \end{tabular}
    \caption{Vlastnosti filtrů medián a BES}
    \end{center}
\end{table} 
    
    Z tabulky je vidět, že fitry vykazují velkou odolnost vůči kontaminaci (v praxi se pohybuje na max 10 \%) a varianci nesnižují nikterak valně -- v případě mediánu z lichého počtu prvků dokonce vůbec. Lepšího snížení variance šumu na úkor nižší odolnosti vůči kontaminaci dosáhneme použítím Walshova seznamu, kde se se variance sníží navíc \emph{až} dvakrát. Může to být i méně, neboť $\Lw \subset \WL$ a při výběru prvku z $\Lw$ se žádné průměrování nekoná a variance se tak nesnižuje. Dále je třeba si uvědomit, že kontaminace \kk~prvků v masce vede k poškození
    \beq
    \mathrm{f_{bad}}(k) = \sum_{i=0}^{k-1}(c-i)
    \eeq
    prvků ve Walshově listu -- jeden kontaminovaný prvek díky průměrování poškodí dohromady $c$ prvků, druhý dalších $c-1$ (jeden z průměrů již poškozený je) atd. Nejvíce problematické jsou průměry dvou extrémních kontaminovaných prvků, které mají přesně polovinu maximální intenzity a tudíž se neusadí na okrajích seřazeného seznamu -- označme je $p_{\frac{1}{2}}$. Těch může být nejvýše $\lfloor\frac{k}{2}\rfloor\cdot\lceil\frac{k}{2}\rceil$. Definujme $\mathrm{f_{bad}^{-1}}(n)$ jako
    \beq
    \mathrm{f_{bad}^{-1}}(n) = \max\bigg\{k \in \Nn_0 \;\bigg\vert\; n \geq \sum_{i=0}^{k-1}(c-i)\bigg\}
    \eeq
    Pokud budeme postupovat jako u filtrů medián a BES, tzn. že může být \emph{poškozeno} \textit{n} prvků seřazeného \emph{seznamu} až do prvního vybraného (buď medián, nebo první kvartil), což odpovídá $\mathrm{f_{bad}^{-1}}(n)$ kontaminovaným prvkům v \emph{masce}, zajistíme maximálně to, že při konstatních datech, kontaminovaných v příslušném poměru\footnote{respektive tak, že v masce bude vždy maximálně $\mathrm{f_{bad}^{-1}}(n)$ kontaminovaných voxelů}, dá filtr konstatntní výstup, tzn. nezahrne žádné poškozené prvky. Nebude-li však vstup konstantní, mohou se poškozené prvky (a zejména prvky $p_{\frac{1}{2}}$) rozprostřít celkem libovolně po celém seznamu a nijak je neodstaníme. V takovém případě ale prohlašme, že poškozené prvky jsou hodně blízko nějakých \bq zdravých\eq ~prvků a tedy jejich výběr pro výpočet výsledku filtru příliš nevadí. Pro zamezení tomu, aby se prvky ze skupinky $p_{\frac{1}{2}}$ (která drží pohromadě) příliš promítly do výsledku, můžeme indexy vybíraných prvků (medián a kvartily) posunout tak, aby vzdálenost mezi nimi byla větší, než počet $p_{\frac{1}{2}}$ prvků (například pomoc zavedení kvazi-mediánu). Toto ale také zanedbáme s ohledem na předchozí úmluvu.

    V tabulce~\ref{tab WBES} uvádíme výsledy Hodges-Lehmanova mediánu a WBES. Kvůli mediánu je opět vhodné, aby Walshův seznam měl sudý počet prvků, to nastane pokud $4|c \,\vee\, 4|(c+1)$, protože Walshův seznam má ${c+1 \choose 2} = \frac{c(c+1)}{2}$ prvků. Masky tedy použijeme stejné jako na obrázku~\ref{obr masky}, tentokrát ovšem s centrálním voxelem.

\begin{table}[h]
    \hspace{-0.6cm}
    \begin{tabular}{lllllll}
      \toprule
      \multirow{4}{*}{Filtr$_{\mathrm{\substack{kapacita\\ masky}}}$} &délka&max konta-& max konta- & odolnost    & snížení    & snížení varia- \\
                                        &Walshova& minovaných & minovaných & proti       & variance   & nce šumu      \\
                                        &seznamu& prvků       & voxelů v   & kontami-    & šumu       & čistým prů-     \\
                                        &       & ve WL ($n$) & masce $\mathrm{f_{bad}^{-1}}(n)$ & naci &  & měrováním  \\
      \midrule
      H-L med$_{\mathrm{7}}$            & 28  & 13  & 2                 & 28,6 \%       & $1/2$ až $1/4$    & $1/7$ \\
      H-L med$_{\mathrm{19}}$           & 190 & 94  & 5                 & 26,3 \%       & $1/2$ až $1/4$    & $1/19$ \\
      H-L med$_{\mathrm{27}}$           & 378 & 188 & 8                 & 29,6 \%       & $1/2$ až $1/4$    & $1/27$  \\
      WBES$_{\mathrm{7}}$               & 28  & 6   & 0 (!)             & 0 \%          & $1/4$ až $1/7$    & $1/7$ \\
      WBES$_{\mathrm{19}}$              & 190 & 47  & 2                 & 10,5 \%       & $1/4$ až $1/8$    & $1/19$ \\
      WBES$_{\mathrm{27}}$              & 378 & 94  & 3                 & 11,1 \%       & $1/4$ až $1/8$    & $1/27$ \\
      \bottomrule
    \end{tabular}
    \caption{Vlastnosti filtrů H-L medián a WBES}
\end{table}\label{tab WBES}

     Vidíme, že zvláště WBES$_{\mathrm{19,27}}$ se odolností proti kontaminaci blíží do teoretických požadovaných mezí a snížením variance už se dostává zhruba na 1/4 až 1/3 výkonu klasického průměru. Zajímavá je situace u WBES$_{\mathrm{7}}$, kde je maska tak malá, že filtr, tak jak je navržen, podle naší úmluvy odolný proti kontaminaci není. To můžeme zlepšit tak, že místo prvního a třetího kvartilu vybereme prvky o jedna blíže ke středu pole -- pokozených pak může být 7, což odpovídá jednomu kontaminovanému v masce a 14,3\% odolnosti. Navíc WBES$_{\mathrm{7}}$ může být ve snižování variance šumu stejně dobrý jako odpovídající průměrový filtr (1/8 dosáhnout nelze, protože máme jen 7 vstupů).
     
     \vspace{0.5cm}
     
     V praxi však měříme kontaminaci pro celý obraz a ne na úrovni jednostlivých voxelů. Může se tedy stát, že při 10\% kontaminaci \emph{celého obrazu} nebude výsledek filtru WBES$_{\mathrm{27}}$ příliš dobrý, protože u mnoha voxelů bude kontaminováno více než 10,5 \% okolních voxelů (kontaminované voxely mohou být libovolně rozmístěny) a výstupní voxel bude poškozený -- což ale odpovídá vlastnostem filtru. V odolnosti proti kontaminaci je tudíž dobré mít oproti teoretickým výsledkům nějakou rezervu.
     
     U agresivnějších šumů (velká kontaminace, velký rozptyl gaussovského šumu) a filtrů na bázi Walshova seznamu navíc nestačí malá maska s poloměrem 1, neboť průměr z tak malého okolí subjektivně nerozpoznáme. Větší maska (poloměr 2) již funguje lépe, ale vzhledem k strmě rostoucímu počtu prvků Walshova seznamu je filtr již příliš časově náročný. Alternativou by mohla být kombinace mediánu (H-L mediánu) na malé masce, který odstraní kontaminaci, a nějakého klasického filtru na bázi aritmetického průměru s větší maskou, který se postará o gaussovský šum. Příklady v příloze~\ref{příloha obrázky}.

\section{Urychlení výpočtů na GPU}

    \subsection{Testovací sestava a data}
    
    \subsubsection{Data}
        
        Jako testovací obraz byl použit 3D scan mozku získaný metodou SPECT\footnote{Single Photon Emission Computed Tomography} o rozměrech $79 \times 95 \times 69$ voxelů v odstínech šedi (0-255) uložený ve formátu \Analyze. Pomocí funkce {\tt AddNoise(...)} byl k těmto datům pro demonstraci odstraňování šumu přidám kontaminovaný gaussovský šum s nastavitelnými parametry (viz příloha~\ref{příloha obrázky}).
        
    \subsubsection{Testovací sestava}
    
        Pro testování filtrů na CPU byla použita sestava: Intel Core$^{\mathrm{\textsc{tm}}}$ 2 DUO @ 1,866 GHz, 2 GB RAM. Aplikace běžela ovšem pouze v jednom vlákně.
        
        GPU byla použita následující: NVIDIA GeForce GTX8800, ???????
    
    \subsection{Výsledky}
    
    V tabulkách \ref{výsl 1} a \ref{výsl 2} uvádíme výsledky pro sedm filtrů, u každého jsme zvolili dvě v praxi používané masky. Měření bylo prováděno pomocí systémového high-performance čítače běžícího na frekvenci CPU (1,866 GHz), každý filtr byl spuštěn 20krát za sebou na CPU a 20krát na GPU, pokaždé na stejných vstupních datech. \note{zdůrazňovat, jak se to na CPU houpe?... dát méně cifer?}
    
\begin{table}[h]
    \begin{center}
    \begin{tabular}{lD{,}{,}{-1}D{,}{,}{-1}D{,}{,}{-1}}
      \toprule
      Filtr$_{\mathrm{kapacita}}$ & \multicolumn{1}{c}{CPU (ms)} & \multicolumn{1}{c}{GPU (ms)} & \multicolumn{1}{c}{Urychlení} \\
      \midrule
      \multicolumn{4}{c}{{\tt unsigned char} (0-255)}  \vspace{0.1cm} \\
      Eroze$_{18}$      & 32,30 & 1,37 & 23,6 \\
      Dilatace$_{18}$   & 32,75 & 1,39 & 23,6 \\
      Hrany$_{18}$      & 58,25 & 1,34 & 43,5 \\
      Eroze$_{26}$      & 44,33 & 1,49 & 29,8 \\
      Dilatace$_{26}$   & 44,86 & 1,49 & 30,1 \\
      Hrany$_{26}$      & 80,83 & 1,59 & 50,8 \\
      \midrule
      \multicolumn{4}{c}{{\tt unsigned int} (0-65535)}  \vspace{0.1cm} \\
      Eroze$_{18}$      & 36,81 & 1,39 & 26,5 \\
      Dilatace$_{18}$   & 32,21 & 1,40 & 23,0 \\
      Hrany$_{18}$      & 57,41 & 1,32 & 43,5 \\
      Eroze$_{26}$      & 50,01 & 1,62 & 30,9 \\
      Dilatace$_{26}$   & 50,21 & 1,62 & 31,0 \\
      Hrany$_{26}$      & 78,91 & 1,56 & 50,6 \\
      \bottomrule
    \end{tabular}
    %\caption{Vlastnosti filtrů H-L medián a WBES}
    \end{center}
\end{table}\label{výsl 1}

\begin{table}[h]
    \begin{center}
    \begin{tabular}{lD{,}{,}{-1}D{,}{,}{-1}D{,}{,}{-1}}
      \toprule
      Filtr$_{\mathrm{kapacita}}$ & \multicolumn{1}{c}{CPU (ms)} & \multicolumn{1}{c}{GPU (ms)} & \multicolumn{1}{c}{Urychlení} \\
      \midrule
      \multicolumn{4}{c}{{\tt unsigned char} (0-255)}  \vspace{0.1cm} \\
      Medián$_{19}$     & 215,18 & 3,62     & 59,4 \\
      BES$_{19}$        & 263,13 & 4,98     & 52,8 \\
      H-L Medián$_{19}$ & 1683,64 & 344,81  & 4,9 \\
      WBES$_{19}$       & 2498,99 & 465,10  & 5,4 \\
      Medián$_{27}$     & 334,62 & 9,44     & 35,5 \\
      BES$_{27}$        & 376,58 & 10,17    & 37,0 \\
      H-L Medián$_{27}$ & 3159,28 & 901,41  & 3,5 \\
      WBES$_{27}$       & 4688,86 & 1125,30 & 4,2 \\
      \midrule
      \multicolumn{4}{c}{{\tt unsigned int} (0-65535)}  \vspace{0.1cm}\\
      Medián$_{19}$     & 206,71 & 5,96     & 34,7 \\
      BES$_{19}$        & 274,45 & 6,45     & 42,6 \\
      H-L Medián$_{19}$ & 1658,44 & 434,21  & 3,8 \\
      WBES$_{19}$       & 2571,52 & 520,41  & 4,9 \\
      Medián$_{27}$     & 315,32 & 25,25    & 12,5 \\
      BES$_{27}$        & 390,39 & 10,96    & 35,6 \\
      H-L Medián$_{27}$ & 3113,22 & 1332,64 & 2,3 \\
      WBES$_{27}$       & 4755,84 & 1339,52 & 3,6 \\
      \bottomrule
    \end{tabular}
    %\caption{Vlastnosti filtrů H-L medián a WBES}
    \end{center}
\end{table}\label{výsl 2}
    
