% -*-coding: utf-8 -*-

    % Lukasiewiczova algebra s odmocniou
            % Konstrukce matematického modelu
                % Svaz
                % Lukasiewiczova algebra
                % Lukasiewiczova algebra s odmocninou
            % Fuzzy-logická funkce
            % Používané operátory
        % Citlivost
            % Přesnost celočíselné aritmetiky
        % Reprezentace obrazových dat
            % Maska
        % Filtry
            % Typologie filtrů
                % Morfologické
                % Statistické
            % Citlivost filtrů
            % Porovnání s lineárními filtry
%==============================================================
        \note{fuzzy -- dvouhodnotová logika je málo - příliš velká ztráta informace, struktura logiky je vhodná pro analýzu vlastností, umožňuje redukovat
        magické konstanty a zároveň je dostačující pro většinu praktických úloh}


\section{\L ukasiewiczova algebra s odmocninou}
    % bez odmocniny bude množina filtrů omezená (uzavření-otevření nic nedělá) -- vyhnout se přímo trvzení o konečnosti, nekonečnosti
                % jen svaz s neg, V, A má 2^(2^n) operací
                %%%% z residua a negace
        % obohatit systém tak, aby se dala snížit citlivost a dělat průměry -- prostě co ejmenší množina operací, se kterými už uděláme, co potřebujeme
        % odmocnina jde vždy propasovat dolů, jde jí předejít vydělením vstupů mocninou 2ky (2a, 4a udělám pomocí krát)

        % porovnání s lineárními filtry: (Gauss, atd)
            % lineární jsou lepší na gaussovský šum
            % fuzzy umožňuje ubrat předpoklady o vstupu (druh šumu, SMĚS ŠUMŮ -- KOSMICKÉ ZÁŘENÍ VE SPECT...)
    % obecně robustní filtry (dá se udělat fuzzy k-tý prvek)
    % při algoritmickém generování sítí, instantním prohlížení a filtrace
    %=============================================================================================================================

    V této části zkonstruujeme matematický model, který bude snadno aplikovatelný na obrazová data ve formě, jak s nimi pracuje počítač a zároveň nám poskytne solidní základ pro konstrukci filtrů. Budeme postupovat podobně jako v \cite{MajerovaPhD} a ukážeme, že \L ukasiewiczova algebra s odmocninou je vhodným kandidátem. Na model přitom klademe následující poždavky:
    \begin{itemize}
      \item maximální jednoduchost matematických operací
      \item \note{struktura (mnohahodnotové) logiky}
      \item omezená citlivost (zbavování šumu)
      \item neomezená množina filtrů, které je možno zkonstruovat
      \item přímá aplikovatelnost na obrazová data v odstínech šedi
      \item možnost konstrukce robustních\footnote{tzn. do výstupu filtru se (příliš) nepromítnou okrajové extrémní hodnoty ze vstupu} nízkofrekvenčních filtrů (zbavení šumu)
      \item možnost konstrukce filtrů na bázi aritmetického průměru \note{nekříží se s poždavkem robustnosti?}
      \item možnost konstrukce vysokofrekvenčních filtrů (detektory hran, segmentace)
    \end{itemize}
    Z poždavku jednoduchosti je jasné, že budeme postupovat od základních matematických struktur a ty postupně obohacovat o další operace.

    \subsection{Matematický model}

    Obraz je v počítači reprezentován jako sada diskrétních hodnot omezených v nějakém intervalu (0.0 až 1.0, 0 až 255...). Struktura musí být tedy uzavřená vůči vlastním operacím, mít jako nosič omezenou množinu a neměla by jí vadit ani diskretizace této množiny.

    \begin{define}\label{svaz}
    Algebru $S = (M,\wedge,\vee)$ se dvěma binárními operacemi takovými, že platí:
    \begin{align}
    a \wedge b &= b \wedge a, & a \vee b &= b \vee a &&\text{\rl komutativní zákon\rr} \\
    (a \wedge b) \wedge c &= a \wedge (b \wedge c), & (a \vee b) \vee c &= a \vee (b \vee c) &&\text{\rl asociativní zákon\rr} \\
    a \wedge (b \vee a) &= a,& a \vee (b \wedge a) &= a &&\text{\rl zákon absorpce\rr}
    \end{align}
    nazýváme \textbf{svazem}. Operace $\wedge$ \rl\textbf{průsek}\rr a $\vee$ \rl\textbf{spojení}\rr dodefinujeme v souladu s teorií:
    \begin{align}
    a \wedge b &= \min(a,b), \\
    a \vee b &= \max(a,b).
    \end{align}
    \end{define}

    Svaz nám bez dalšího obohacení umožňuje konstruovat základní morfologické filtry jako eroze a dilatace\footnote{tyto a všechny další filtry budou vysvětleny v sekci~\ref{Filtry}, zde slouží pouze pro ilustraci} a je nas ním možno zkonstruovat libovolnou pořadovou statistiku (k-tý prvek) -- základ mnoha robustních filtrů ostraňujících šum:

    \begin{theo}\label{theo k-tý prvek}
      Nechť $x_1,x_2,...,x_n \in M$, dále buď $l = n-k+1$, $C^{l}_n$ oindexovaná množina všech l-tic z \^n a $c^i_j$, kde $c^i \in C^{l}_n$, j-tý prvek i-té l-tice. Pak
      \[
        a_k = \bigwedge_{i = 1}^{{l}\choose{n}}\left( \bigvee_{j = 1}^{l} x_{c_{j}^i} \right)
      \]
      je k-tá pořadová statistika souboru $(x_1,x_2,...,x_n)$ \rl první prvek je maximum\rr.
    \end{theo}
    \begin{proof}
      Vnitřní operace vybere postupně ze všech l-tic maximum. Nejmenšího maxima se dosáhne právě tehdy, chybí-li v l-tici největších $k-1$ prvků. Tímto maximem bude tedy právě k-tý prvek.
    \end{proof}

    Bohužel už medián ze sudého počtu prvků je nerealizovatelný (chybí nám aritmetický průměr), stejně jako jakýkoliv vysokofrekvenční filtr\footnote{například detektor hran} (chybí nám rozdíl). Z povahy svazových operací je navíc zřejmé, že nikdy nezískáme hodnotu, která se neobjevila na vstupu.

    Abychom získali možnost odčítat, obohatíme svaz o dvě binární operace, z nichž se dá operace rozdílu odvodit (viz \ref{operátory}):

    \begin{define}\label{def residuovaný svaz}
    Buď $(M,\wedge,\vee)$ je svaz s nejmenším prvkem 0 a největším 1, $\otimes$ \rl \textbf{součin}, nebo \textbf{t-norma}\rr je binární, asociativní a komutativní operace a $\rightarrow$ \rl \textbf{residuum}\rr je binární operace. Navíc platí:
    \begin{align}
    &x \otimes 1 = x  &&(\forall x \in M)\\
    &x \otimes y \leq z \Leftrightarrow y \rightarrow z \geq x &&(\forall x,y,z \in M)\label{Galoisova koresp}
    \end{align}
    Pak strukturu $(M,\wedge,\vee,\otimes,\rightarrow,0,1)$ nazýváme \textbf{residuovaný svaz}.
    \end{define}

    Galoisova korespondence \eqref{Galoisova koresp} přiřazuje každému součinu právě jedno residuum \note{zdroj něco z Majerové?}, k plnému popisu operací tedy stačí definovat pouze součin. Za ten zvolíme \L ukasiewiczovu t-normu:

    \begin{define}\label{LA}
    Residuovaný svaz $(\LL,\wedge,\vee,\otimes,\rightarrow,0,1)$ s operacemi $\otimes$ a $\rightarrow$ definovanými v souladu s \eqref{Galoisova koresp} jako
    \begin{align}
    x \otimes y &= \max(x+y-1,0) \\
    x \rightarrow y &= \min(1,1-x+y)
    \end{align}
    a nosičem $\LL = [0,1]$ \rl 0,1 zde můžeme brát opět jako obecný nejmenší a největší prvek, interval později ztotožníme s rozsahem intenzit pixelů v obraze\rr \note{co \ref{def obraz}? je to takhle elegantní?} nazýváme \textbf{\L ukasiewiczova algebra} \textup{(\L A)}.
    \end{define}

    \note{0,1 zde představuje jen největší a nejmenší prvek -- je nutné to zmínit?}
    \note{Proč potřebujeme MV-algebru? Gougen, Godel ani Yager nejsou --- říká,že neg neg x je x, což je slušné chování}

    Existují i jiné t-normy (\cite{MajerovaPhD},\cite{Bělíček}), ty ovšem nemají vlastnosti mnohahodnotové logiky \cite{MajerovaPhD}\note{původnější zdroj?}, nebo jsou příliš výpočetně náročné. Poslední obohacení, které provedeme, nám konečně umožní průměrovat:

    \begin{define}\label{LAsqrt}
    \L ukasiewiczovu algebru $(\LL,\wedge,\vee,\otimes,\rightarrow,\sqr,0,1)$ obohacenou o unární operaci \textbf{odmocnina} pro kterou platí:
    \beq
    \sqr(x)\otimes\sqr(x) = x
    \eeq
    nazýváme \textbf{\L ukasiewiczova algebra s odmocninou} \textup{(\LAsq)}.
    \end{define}

    Snadno se přesvědčíme, že
    \beq
    \sqr(x) = \frac{1+x}{2}
    \eeq
    Aritmetický průměr \emph{dvou} pak můžeme realizovat jako
    \beq
    \sqr(x)\otimes \sqr(y) = \max\left(\frac{1+x}{2}+\frac{1+y}{2}-1,0\right) = \max\left(\frac{x+y}{2},0\right) = \frac{x+y}{2}
    \eeq
    což je vlastně geometrický průměr uvnitř \LAsq. Analogicky by bylo možné definovat libovolnou celočíselnou odmocninu, která by nám umožnila průměrovat libovolný počet hodnot. Filtry založené na průměru mnoha hodnot již však nejsou robustní, a tak se schválně omezíme pouze na druhou odmocninu. I s těmito omezeními je však možné zkonstruovat celou třídu tzv. \emph{diadických filtrů} -- u nich je součet vah vstupů roven $2^n$ a výstupem je vážený průměr. \note{váhy libovolné??, průměr z $2^n$ umíme}

    \subsection{Fuzzy-logické funkce}

    Základní operace pro tvorbu filtrů tedy máme. Vzhledem k tomu, že se pohybujeme v praktických vodách, a při výpočtech filtrů bude hrát hlavní roli čas, těžko si představit, že například k-tou pořadovou charakteristiku budeme implementovat tak, jak je popsána v definici~\ref{k-tý prvek}. Přijměme tedy úmluvu, že matematický model nám řekne, jaké operace/funkce/filtry jsou přípustné, ale při implementaci budeme chtít, aby pouze výsledek (nikoliv samotný algoritmus) byl v souladu s teorií (čímž získáme prostor pro kýžené optimalizace). Pro další postup bude tedy vhodné rozlišit, jaké funkce do modelu patří a co nikoliv.

    \begin{define}\label{def FLF}
    Buď $x \in \LL$ proměnná, $a \in \LL$ konstanta. Pak množinu fuzzy-logických funkcí \textup{(FLF)}, jakožto zobrazení $\LL^k \rightarrow \LL, k \in \mathbb{N}_0$, definujeme indukčním pravidlem:
    \[
    \mathrm{FLF} = x \mid a \mid \f \wedge \g \mid \f \vee \g \mid \f \otimes \g \mid \f \rightarrow \g \mid \sqr(f)
    \]
    kde $\f,\g$ jsou již \textup{FLF}.
    \end{define}

    \subsection{Používané operátory}\label{operátory}

    Následuje tabulka běžných fuzzy-logických operátorů včetně definic a výsledných hodnot v \LAsq. Všechny operátory vzniknou pouze složením základních operátorů a jsou tedy FLF.

    \begin{table}[h]
    \begin{center}
    \begin{tabular}{llll}
      \toprule
      Operace & Název & Odvození & Význam v \LAsq \\
      \midrule
      $x \vee y$            & spojení                   & definováno & $\max(x,y)$ \\
      $x \wedge y$          & průsek                    & definováno & $\min(x,y)$ \\
      $x \otimes y$         & součin (t-norma)          & definováno & $\max(x+y-1,0)$ \\
      $x \rightarrow y$     & residuum                  & definováno & $\min(1-x+y,1)$ \\
      $ \neg x$             & negace                    & $x \rightarrow 0$ & $1-x$ \\
      $x \leftrightarrow y$ & ekvivalence (biresiduum)  & $(x \rightarrow y) \wedge (y \rightarrow x)$ & $1-|x-y|$ \\
      $x \circ y$           & vzdálenost                & $\neg (x \leftrightarrow y)$ & $|x-y|$ \\
      $x \oplus y$          & součet                    & $\neg (\neg x \otimes \neg y)$ & $\min(x+y,1)$ \\
      $x \ominus y$         & rozdíl                    & $x \otimes \neg y$ & $\max(x-y,0)$ \\
      $x \odot n$           & celočíselný násobek       & $\underbrace{x \oplus x \oplus ... \oplus x}_n, n \in \mathbb{N}$ & $\min(nx,1)$ \\
      $x^n$                 & celočíselná mocnina       & $\underbrace{x \otimes x \otimes ... \otimes x}_n, n \in \mathbb{N} $ & $\max(nx-n+1,0)$ \\
      \bottomrule
    \end{tabular}
    \caption{Odvozené operátory v \LAsq}
    \end{center}
    \end{table}

\section{Citlivost}

\begin{define}\label{def citlivost}
  Buď $\varphi : \LL^n \rightarrow \LL$ \textup{FLF} a $\xx,\yy \in \LL^n$. Pak
  \beq
  \lambda_\varphi = \max_{\xx \neq \yy}\frac{\varphi(\xx) \circ \varphi(\yy)}{\sum_{i=1}^n |x_i - y_i|}\label{citlivost}
  \eeq
  nazveme \textbf{citlivostí} funkce $\varphi$.
\end{define}

Pro další úvahy je dobré znát horní meze citliosti běžných fuzzy-logických operátorů. Ty přehledně shrnuje tabulka \ref{tabulka max citlivosti} a jsou dokázány v \cite{MajerovaPhD}. Buď $\xx \in \LL^n ,n \in \mathbb{N} ,\f ,\g : \LL^n \rightarrow \LL $ FLF \note{jak to hodit k tabulce}:

\begin{table}\label{tabulka max citlivosti}
    \begin{center}
    \begin{tabular}{llp{1cm}ll}
      \toprule
      Operace & $\lambda_{\max}$ && Operace & $\lambda_{\max}$\\
      \midrule
      $\f(\xx)$                         & $\lambda_\f$  && $\f(\xx) \vee \g(\xx)$            & $\max(\lambda_\f,\lambda_\g)$  \\
      $\g(\xx)$                         & $\lambda_\g$  && $\f(\xx) \wedge \g(\xx)$          & $\max(\lambda_\f,\lambda_\g)$  \\
      $\f(\xx) \otimes \g(\xx)$         & $\lambda_\f+\lambda_\g$ && $ \neg \f(\xx)$         & $\lambda_\f$    \\
      $\f(\xx) \rightarrow \g(\xx)$     & $\lambda_\f+\lambda_\g$ && $\sqr(\f(\xx))$         & $\lambda_\f/2$ \\
      $\f(\xx) \leftrightarrow \g(\xx)$ & $\lambda_\f+\lambda_\g$ && $\f(\xx) \oplus \g(\xx)$& $\lambda_\f+\lambda_\g$    \\
      $\f(\xx) \circ \g(\xx)$           & $\lambda_\f+\lambda_\g$ &&$\f(\xx) \ominus \g(\xx)$& $\lambda_\f+\lambda_\g$    \\
      $\f(\xx) \odot n$                 & $n\lambda_\f$ &&  $(\f(\xx))^n$                    & $n\lambda_\f$ \\
      \bottomrule
    \end{tabular}
    \caption{Horní meze citlivosti operátorů v \LAsq}
    \end{center}
\end{table}

Citlivost $\lambda_\varphi = 0$ znamená, že funkce $\varphi(\xx) = konst, \, \forall \xx \in \LL^n$ (v čitateli \eqref{citlivost} musí být 0 $\forall \xx$), což odpovídá ztátě veškeré informace. Nízkofrekvenční filtry mají citlivost obecně v intervalu $(0,1]$, vysokofrekvenční pak v intervalu $(1,\infty)$. Zjednodušeně řečeno nízká citlivost způsobuje vymizení detailů, zatímco vysoká může mít za následek vznik nových artefaktů v obraze, vše ale záleží na konkrétním filtru \note{(ke schválení)}. Z tabulky je dále zřejmé, proč jsme zaváděli odmocninu -- ta jako jediná dokáže citlivost snížit, nebo ji spíše při skládání operací udržet v požadovaném intervalu. \note{dokazovat horní mez citlivosti pro  k-tý prvek, nebo to nechat až na filtry}


\section{Reprezentace obrazových dat}
% greyscale
% 2D, 3D

Obrazová data budeme chtít zpracovat pomocí FLF, tzn. musíme přiřadit vstupům funkce hodnoty konkrétních pixelů. Předem se omezíme pouze na obraz v odstínech šedi (medicínská data) a pro přehlednost na 2 dimenze (analogie se 3D je zřejmá a podstatně to zjednoduší zápis).

\begin{define}\label{def obraz}
  \textbf{Obraz} o šířce $w$ a výšce $h$ ztotožníme s maticí $\PP \in \LL^{w,h}$, kde $\LL$ je interval zahrnující všechny přípustné intenzity v obraze. Dále pixel $(\PP)_{i,j}, \,i \in \widehat w, \,j \in \widehat h$ označíme jako $x_{i,j}$.
\end{define}

    \subsection{Lokální zpracování obrazu}
    % předem definované pixely vs watershed
    Všechny dále popsané filtry zpracovávají obraz lokálně \note{podle mě neplyne z FLF, prože to by omezilo nanejvýš počet vstupů}, to znamená, že do výsledné intenzity pixelu se promítnou pouze intezity několika přesně definovaných okolních pixelů. Jako opačný příklad může sloužit složitější segmentační filtr \emph{rozvodí} (Watershed, \cite{Charypar}), u kterého nemůžeme předem říci, které pixely budou mít vliv na výsledek, a dokonce ani kolik jich bude.

    \begin{define}\label{def okolí}
      Množinu pixelů
      \beq
      \NN_{R}(x_{i,j}) = \Big\{ x_{k,l} \in \PP \;\Big\vert\; |k-i| \leq R \wedge |l-j| \leq R\, \, k,l \in \mathbb{Z} \Big\}
      \eeq
      kde $R\in \mathbb{N}_0, \,x_{i,j} \in \PP$ nazýváme \textbf{okolím pixelu $x_{i,j}$} o poloměru $R$ a $x_{i,j}$ nazýváme \textbf{centrální pixel}. Okolí představuje čtverec pixelů $2R+1 \times 2R+1$, který můžeme přeznačit, a po řádcích uspořádat do \textbf{vektoru okolí} $\Nb = (x_1,x_2,...,x_{(2R+1)^2})$.
    \end{define}

    Z předchozí definice je patrné, že bude problém s okraji obrázku (protože pouze centrální pixel musí ležet uvnitř obrázku). Aby nedocházelo k degeneraci okolí, dodefinujeme intenzity pixelů vně obrázku nějakým zvoleným způsobem \cite{MajerovaPhD}\note{dát obrázky?}: buď konstatní nulou, hodnotou nejbližího krajního pixelu, periodickým opakováním obrázku, nebo jeho ozrcadlením přes okraj (v rozích se zrcadlí dvakrát). Nejvhodnější způsob závisí na mnoha okolnostech -- od volby filtrů až po charakter samotných dat.

    \subsection{Maska}

    Abychom získali větší variabilitu, ohodnotíme každý pixel z okolí vahou:

    \begin{define}\label{def maska}
      \textbf{Maskou} s poloměrem $R$ nazveme čtvercovou matici $\MM_R \in (\mathbb{Z}^+)^{2R+1,2R+1}$ s prvky $m_{i,j}$. Číslo
      \beq
      k = \sum_{i,j=1}^r m_{i,j}
      \eeq
      nazýváme \textbf{kapacita} masky. Masku můžeme taktéž po řádcích přeznačit a vytvořit \textbf{vektor vah} $\mathrm{W} = (w_1,w_2,...,w_{(2R+1)^2})$.
    \end{define}

    Binární masku (\note{?, nebo obecně nenulovou část masky?}) v matematické morfologii (viz sekce \ref{Typologie}) označujeme také jako \emph{strukturní element}. Ve většině aplikací se setkáme pouze s relativně malým $R$ (řekněme $R \leq 3$, \cite{MajerovaPhD}), protože s rostoucím poloměrem masky jednak velmi rychle roste náročnost zpracování (minimálně $O(n^2)$, potažmo $O(n^3)$ ve 3D krát náročnost filtru) a druhak u nízkofrekvnčních filtrů je efektivnější zmenšit celý obraz a použít menší masku (zvláště, pokud je granularita šumu větší než 1 pixel). U vysokofrekvenčních filtrů nemá větší maska smysl vůbec, neboť tím dochází k nežádoucímu rozmazání.

    \subsection{Seznam}\note{přesunout do filtrů?}

    Nyní můžeme získat \emph{vážený seznam vstupních hodnot} pro zpracování filtrem jednoduchým přiložením středu masky ($m_{R+1,R+1}$) na centrální pixel okolí $\NN_{R}(x_{i,j})$:

    \begin{define}\label{def vážený seznam}
    Buď $\MM_R$ maska s kapacitou $k$ a poloměrem $R$, $\mathrm{W}$ její vektor vah a $\Nb$ vektor okolí. Pak k-prvkový seznam
    \beq
    \Lw = (\underbrace{x_1,...,x_1}_{w_1},\underbrace{x_2,...,x_2}_{w_2},...,\underbrace{x_{(2R+1)^2},...,x_{(2R+1)^2}}_{w_{(2R+1)^2}})
    \eeq
    kde $x_m \in \Nb$, $w_m \in \mathrm{W}$, nazvame \textbf{váženým seznamem} hodnot z okolí $\NN_{R}(x_{i,j})$.
    \end{define}

    Pro některé filtry se používá ještě takzvaný \emph{Walshův seznam}:

    \begin{define}\label{def Walshův seznam}
    Seznam o $k+1 \choose 2$ prvcích
    \beq
    \WL = \Big( \frac{x_m + x_n}{2} \Big\vert x_m,x_n \in \Lw, \,m,n \in \widehat{k} \Big)
    \eeq
    tedy obsahující aritmetické průměry všech dvojic z $\Lw$ a samotný seznam $\Lw$ nazveme \textbf{Walshův seznam}
    \end{define}

    Všechny prvky $\WL$ jsou triviálně FLF.


% MASKA:
% co to je, váhy (opakovaný výběr), tvoří se z ní seznam jako základní prvek k zpracování
% okraje
% váhy mohou jít proti robustnosti, hrání s váhami

\section{Filtry}\label{Filtry}
    % něco se dělá nad jedním nebo někalika seznamy a je realizovatelné pomocí FLF, je filtr
    % omezujeme použití konstant -- ztrácí se robustnost, de facto je to přepoklad o vstupu

    Nyní už máme dostatečné prostředky k tomu, abychom popsali postup filtrace obrazu:
    \begin{define}\label{def filtr}
      \textbf{Filtrací} zdrojového obrazu $\PP_1$ na výsledný obraz $\PP_2$ stejné velikosti pomocí filtru $\varphi \in \mathrm{FLF}$ a $n$ masek $\MM_{R}^{(1)},\MM_{R}^{(2)},...,\MM_{R}^{(n)}$ rozumíme následující postup:
      \begin{enumerate}
      \item z okolí zdrojového pixelu $x_{i,j}^{(\PP_1)}$ vytvoříme pomocí masek $\MM_{R}^{(r)}$ vážené seznamy hodnot $\Lw^{(r)} \,\, \forall r \in \widehat n$
      \item pokud to filtr vyžaduje, vytvoříme z $\Lw^{(r)}$ Walshovy listy pro požadovaná $r$
      \item příslušné seznamy použijeme jako vstup pro $\varphi$
      \item výsledek zapíšeme do odpovídajícího cílového pixelu $x_{i,j}^{(\PP_2)}$
      \end{enumerate}
      \textbf{Filtrem} tedy rozumíme libovolnou funkci $\varphi \in \mathrm{FLF}, \,\varphi : \LL^s \rightarrow \LL$, kde $s$ je součet počtu prvků všech seznamů použitých na vstupu $\varphi$.
    \end{define}
    
    Ve většině případů bude $n = 1$, ale mohou se vyskytnout i fitry využívající např. více asymetrických masek.
    
    \subsection{Typologie filtrů}\label{Typologie}       % (implementační hledisko)
    
    Hledisek, podle kterých by se daly filtry roztřídit je více. Jedno, souvidející se tím, jaké frekvence se výsledném obraze zachovají, jsme již zmínili. Vzhledem k tomu, že se dále v práci chceme věnovat hlavně implementaci filtrů, zmíníme ještě jedno rozdělení, které určuje, jak filtr zachází se vstupními daty -- totiž rozdělení na \emph{morfologické} a \emph{statistické} filtry.
    
        \subsubsection{Morfologické}
        Jak název napovídá, do této kategorie patří filtry snažící se nějak postihnout či zvýraznit charakter tvarů v obraze. Základ tvoří dva \emph{morfologické operátory} \emph{eroze} a \emph{dilatace}:
        
        \begin{define}\label{de eroze dilatace}
          Buď $\xx$ seznam délky $k$ vytvořený maskou $\MM_R$. Pak filtr
          \beq
          \EE(\xx) = \bigwedge_{i=1}^k x_i
          \eeq
          nazveme \textbf{erozí} s maskou $\MM_R$ a filtr
          \beq
          \DD(\xx) = \bigvee_{i=1}^k x_i
          \eeq
          nazveme \textbf{dilatací} s maskou $\MM_R$.
        \end{define}
        
        Filtry jsou komplementární, proto popíšeme pouze erozi: původ jejího názvu je nejvíce patrný při aplikaci na binární (černo-bílý) obraz, kde způsobuje zmenšení bílých oblastí, odstranění bílých detailů (světlé části šumu typu \emph{sůl a pepř}) a rozšíření tmavých \bq trhlin\eq. Na obraze v odstínech šedi dochází obecně ke ztmavnutí, které se nejvíce projeví na ostrých hranách, jinak je efekt srovnatelný s černo-bílým protějškem.
        
        Z povahy operací je zřejmé, že váhy větší než 1 se neprojeví a stačí uvažovat binární masku (\emph{strukturní element}). To je obecná vlastnost i dalších morfologických filtrů, které nejsou postavené přímo na erozi a dilataci, protože jejich výsledky mají vždy charakter maxima nebo minima z několika hodnot.\note{Tvar masky zde má podle mě větší vliv, než u statisktických filtrů (?)}
        
        Další oblíbené morfologické operátory vzniknou právě kombinací těchto dvou:
        
        \begin{define}\label{de eroze dilatace}
          Buď $\xx$ seznam délky $k$ vytvořený maskou $\MM_R$. Pak filtr
          \beq
          \OO(\xx) = \DD(\EE(\xx))
          \eeq
          nazveme \textbf{otevřením} a filtr
          \beq
          \CC(\xx) = \EE(\DD(\xx))
          \eeq
          nazveme \textbf{uzavřením}, přičemž u obou filtrů jsou eroze a dilatace prováděny se stejnou maskou $\MM_R$.
        \end{define}
        
        Filtry odstraní světlé (respektive tmavé) detaily, aniž by zbytek obrázku příliš poškodily, navíc platí \note{zdroj? důkaz?}
        
        \beq
        \OO(\OO(\xx)) = \OO(\xx), \quad \CC(\CC(\xx)) = \CC(\xx)
        \eeq
        
        Jejich kombinací můžeme získat dva jednoduché vyhlazovací filtry $\OO(\CC(\xx))$ a $\CC(\OO(\xx))$ \note{nejsou náhodou stejné? - ověřit}, k vyhlazování a zbavení šumu se ale obecně hodí spíše statistické filtry
        
        Rozdílem dilatace a eroze získáme jednoduchý detektor hran \note{nejmenuje se "základní morfologický operátor"?}, nazývaný také \emph{Minkovského kolása} (pro podobnost klobásy a výsledku filtrování binárního obrazu pomocí masky s velkým poloměrem). Další fuzzy hranové detektory můžeme nalézt např. v \cite{Bělíček}.
        
            % zmínit binární masku jako SE
        \subsubsection{Statistické}
        
        Statistické filtry (přijde dodělat)
        
        -- pracují nad seřazenými seznamy
        
        -- malá citlivost na tvar masky -- jen na její poloměr z důvodu robustnosti
        
        -- robustnost -- moc se nečachruje s váhami, ty jí ubírají
        
        -- (využívají k-tou pořadovou statistiku, z tohoto hlediska by mohly být morfologické pouze podskupinou, ale vzhledem k tomu, že maximum a minimum jde z hlediska implementace zařídit jedndušeji a že se mortf. používají hlavně ke konstrukci detektorů hran (využívají rozdíl), je z nich samostatná skupina -- jde o filozofii)
        
        
    \subsection{Porovnání s lineárními filtry}
    
        úvahy o robustnosti -- nepoužívat příliš váhy
        
        lineární gauss -- na gaussův šum super, ale konstanty v masce vlastně odpovídají prvotnímu předpokladu o obrázku, to robustní fuzzy nedělají, zvládají lépe různé kontaminované šumy... 

        
        
        \subsubsection{Sítě a kompromisní filtrování}
        \note{pouze zmínit v souvislosti s nutností rychlého zpracování -- je nutno algoritmicky vyzkoušet velké množství různých kombinací filtrů}



% Kvalita obrázku:

% TODO: viziualizovat rozdíly mezi filtry, jako "standardy" brát výsledky předchozích prací 