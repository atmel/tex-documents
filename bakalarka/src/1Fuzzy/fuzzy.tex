% -*-coding: utf-8 -*-

    % Lukasiewiczova algebra s odmocniou
            % Konstrukce matematického modelu
                % Svaz
                % Lukasiewiczova algebra
                % Lukasiewiczova algebra s odmocninou
            % Fuzzy-logická funkce
            % Používané operátory
        % Citlivost
            % Přesnost celočíselné aritmetiky
        % Reprezentace obrazových dat
            % Maska
        % Filtry
            % Typologie filtrů
                % Morfologické
                % Statistické
            % Citlivost filtrů
            % Porovnání s lineárními filtry
%==============================================================
        \note{fuzzy -- dvouhodnotová logika je málo - příliš velká ztráta informace, struktura logiky je vhodná pro analýzu vlastností, umožňuje redukovat
        magické konstanty a zároveň je dostačující pro většinu praktických úloh}


\section{\L ukasiewiczova algebra s odmocninou}
    % bez odmocniny bude množina filtrů omezená (uzavření-otevření nic nedělá) -- vyhnout se přímo trvzení o konečnosti, nekonečnosti
                % jen svaz s neg, V, A má 2^(2^n) operací
                %%%% z residua a negace
        % obohatit systém tak, aby se dala snížit citlivost a dělat průměry -- prostě co ejmenší množina operací, se kterými už uděláme, co potřebujeme
        % odmocnina jde vždy propasovat dolů, jde jí předejít vydělením vstupů mocninou 2ky (2a, 4a udělám pomocí krát)

        % porovnání s lineárními filtry: (Gauss, atd)
            % lineární jsou lepší na gaussovský šum
            % fuzzy umožňuje ubrat předpoklady o vstupu (druh šumu, SMĚS ŠUMŮ -- KOSMICKÉ ZÁŘENÍ VE SPECT...)
    % obecně robustní filtry (dá se udělat fuzzy k-tý prvek)
    % při algoritmickém generování sítí, instantním prohlížení a filtrace
    %=============================================================================================================================

    V této části zkonstruujeme matematický model, který bude snadno aplikovatelný na obrazová data ve formě, jak s nimi pracuje počítač a zároveň nám poskytne solidní základ pro konstrukci filtrů. Budeme postupovat podobně jako v \cite{MajerovaPhD} a ukážeme, že \L ukasiewiczova algebra s odmocninou je vhodným kandidátem. Na model přitom klademe následující poždavky:
    \begin{itemize}
      \item maximální jednoduchost matematických operací
      \item \note{struktura (mnohahodnotové) logiky}
      \item omezená citlivost (zbavování šumu)
      \item neomezená množina filtrů, které je možno zkonstruovat
      \item přímá aplikovatelnost na obrazová data v odstínech šedi
      \item možnost konstrukce robustních nízkofrekvenčních filtrů (zbavení šumu)
      \item možnost konstrukce filtrů na bázi aritmetického průměru
      \item možnost konstrukce vysokofrekvenčních filtrů (detektory hran, segmentace)
    \end{itemize}
    Z poždavku jednoduchosti je jasné, že budeme postupovat od základních matematických struktur a ty postupně obohacovat o další operace.

    \subsection{Matematický model}

    Obraz je v počítači reprezentován jako sada diskrétních hodnot omezených v nějakém intervalu (0.0 až 1.0, 0 až 255...). Struktura musí být tedy uzavřená vůči vlastním operacím, mít jako nosič omezenou množinu a neměla by jí vadit ani diskretizace této množiny.

    \begin{define}\label{svaz}
    Algebru $S = (M,\wedge,\vee)$ se dvěma binárními operacemi takovými, že platí:
    \begin{align}
    a \wedge b &= b \wedge a, & a \vee b &= b \vee a &&\text{\rl komutativní zákon\rr} \\
    (a \wedge b) \wedge c &= a \wedge (b \wedge c), & (a \vee b) \vee c &= a \vee (b \vee c) &&\text{\rl asociativní zákon\rr} \\
    a \wedge (b \vee a) &= a,& a \vee (b \wedge a) &= a &&\text{\rl zákon absorpce\rr}
    \end{align}
    nazýváme \textbf{svazem}. Operace $\wedge$ \rl\textbf{průsek}\rr a $\vee$ \rl\textbf{spojení}\rr dodefinujeme v souladu s teorií:
    \begin{align}
    a \wedge b &= \min(a,b), \\
    a \vee b &= \max(a,b).
    \end{align}
    \end{define}

    Svaz nám bez dalšího obohacení umožňuje konstruovat základní morfologické filtry jako eroze a dilatace\footnote{tyto a všechny další filtry budou vysvětleny v sekci~\ref{Filtry}, zde slouží pouze pro ilustraci} a je nas ním možno zkonstruovat libovolnou pořadovou statistiku (k-tý prvek) -- základ mnoha robustních filtrů ostraňujících šum:

    \begin{theo}\label{k-tý prvek}
      Nechť $x_1,x_2,...,x_n \in M$, dále buď $l = n-k+1$, $C^{l}_n$ oindexovaná množina všech l-tic z \^n a $c^i_j$, kde $c^i \in C^{l}_n$, j-tý prvek i-té l-tice. Pak
      \[
        a_k = \bigwedge_{i = 1}^{{l}\choose{n}}\left( \bigvee_{j = 1}^{l} x_{c_{j}^i} \right)
      \]
      je k-tá pořadová statistika souboru $(x_1,x_2,...,x_n)$ \rl první prvek je maximum\rr.
    \end{theo}
    \begin{proof}
      Vnitřní operace vybere postupně ze všech l-tic maximum. Nejmenšího maxima se dosáhne právě tehdy, chybí-li v l-tici největších $k-1$ prvků. Tímto maximem bude tedy právě k-tý prvek. 
    \end{proof}
    
    Bohužel už medián ze sudého počtu prvků je nerealizovatelný (chybí nám aritmetický průměr), stejně jako jakýkoliv vysokofrekvenční filtr\footnote{obsahují například detektory hran} (chybí nám rozdíl). Z povahy svazových operací je navíc zřejmé, že nikdy nezískáme hodnotu, která se neobjevila na vstupu.
    
    Abychom získali možnost odčítat, obohatíme svaz o dvě binární operace, z nichž se dá operace rozdílu odvodit\footnote{viz \ref{operátory}}:
    
    \begin{define}\label{residuovaný svaz}
    Buď $(M,\wedge,\vee)$ je svaz s nejmenším prvkem 0 a největším 1, $\otimes$ \rl \textbf{součin}, nebo \textbf{t-norma}\rr je binární, asociativní a komutativní operace a $\rightarrow$ \rl \textbf{residuum}\rr je binární operace. Navíc platí:
    \begin{align}
    &x \otimes 1 = x  &&(\forall x \in M)\\
    &x \otimes y \leq z \Leftrightarrow y \rightarrow z \geq x &&(\forall x,y,z \in M)\label{Galoisova koresp}
    \end{align}
    Pak strukturu $(M,\wedge,\vee,\otimes,\rightarrow,0,1)$ nazýváme \textbf{residuovaný svaz}. 
    \end{define}
    
    Galoisova korespondence \eqref{Galoisova koresp} přiřazuje každému součinu právě jedno residuum \note{zdroj něco z Majerové?}, k plnému popisu operací tedy stačí definovat pouze součin. Za ten zvolíme \L ukasiewiczovu t-normu:
    
    \begin{define}\label{LA}
    Residuovaný svaz $(\LL,\wedge,\vee,\otimes,\rightarrow,0,1)$ s operacemi $\otimes$ a $\rightarrow$ definovanými v souladu s \eqref{Galoisova koresp} jako
    \begin{align}
    &x \otimes y = \max(x+y-1,0) \\
    &x \rightarrow y = \min(1,1-x+y)
    \end{align}
    a nosičem $\LL = [0,1]$ nazýváme \textbf{\L ukasiewiczova algebra} \textup{(\L A)}.
    \end{define}
    
    \note{Proč potřebujeme MV-algebru? Gougen, Godel ani Yager nejsou --- říká,že neg neg x je x, což je slušné chování}
    
    Existují i jiné t-normy (\cite{MajerovaPhD},\cite{Bělíček}), ty ovšem nemají vlastnost mnohahodnotové logiky (\cite{MajerovaPhD} \note{původnější zdroj?}), nebo jsou příliš výpočetně náročné. Poslední obohacení, které provedeme nám konečně umožní dělat aritmetický průměr dvou hodnot:
    
    \begin{define}\label{LAsqrt}
    \L ukasiewiczovu algebru $(\LL,\wedge,\vee,\otimes,\rightarrow,\sqr,0,1)$ obohacenou o unární operaci \textbf{odmocnina} definovanou jako
    \[
    \sqr(x) = \frac{1+x}{2}
    \]
    nazýváme \textbf{\L ukasiewiczova algebra s odmocninou} \textup{(\LAsq)}.
    \end{define}
    
    Aritmetický průměr \emph{dvou} pak můžeme realizovat jako
    
    \[
    \sqr(x)\otimes \sqr(y) = \max\left(\frac{1+x}{2}+\frac{1+y}{2}-1,0\right) = \max\left(\frac{x+y}{2},0\right) = \frac{x+y}{2}
    \]
    
    Což je vlastně geometrický průměr uvnitř \LAsq. Tímto způsobem můžeme sice průměrovat pouze $2^n, (n \in \mathbb{N})$ hodnot, v praxi se však ukazuje, že pro konstrukci robustních nízkofrekvenčních filtrů je to dostačující a pro vysokofrekvenční filtry průměr příliš nepotřebujeme. \note{zmínit diadický filtr?}

    \subsection{Fuzzy-logické funkce}
    
    Základní operace pro tvorbu filtrů tedy máme. Vzhledem k tomu, že se pohybujeme v praktických vodách, a při výpočtech filtrů bude hrát hlavní roli čas, těžko si představit, že například k-tou pořadovou charakteristiku budeme implementovat tak, jak je popsána v definici~\ref{k-tý prvek}. Přijměme tedy úmluvu, že matematický model nám řekne, jaké operace/funkce/filtry jsou přípustné, ale při implementaci budeme chtít, aby pouze výsledek (nikoliv samotný algoritmus) byl v souladu s teorií (čímž získáme prostor pro kýžené optimalizace). Pro další postup bude tedy vhodné rozlišit, jaké funkce do modelu patří a co nikoliv.
    
    \begin{define}\label{FLF}
    Buď $x \in \LL$ proměnná, $a \in \LL$ konstanta. Pak množinu fuzzy-logických funkcí \textup{(FLF)}, jakožto zobrazení $\LL^k \rightarrow \LL, k \in \mathbb{N}_0$, definujeme indukčním pravidlem:
    \[
    \mathrm{FLF} = x \mid a \mid f \wedge g \mid f \vee g \mid f \otimes g \mid f \rightarrow g \mid \sqr(f) 
    \]
    kde $f,g$ je již \textup{FLF}.
    \end{define}
    
    \subsection{Používané operátory}\label{operátory}
    
    Následuje tabulka běžných fuzzy-logických operátorů včetně definic a výsledných hodnot v \LAsq. Všechny operátory vzniknou pouze složením základních operátorů a jsou tedy FLF.
    
    \begin{table}[h]
    \begin{center}
    \begin{tabular}{llll}
      \toprule
      Operace & Název & Odvození & Význam v \LAsq \\
      \midrule
      $x \vee y$            & spojení                   & definováno & $\max(x,y)$ \\
      $x \wedge y$          & průsek                    & definováno & $\min(x,y)$ \\
      $x \otimes y$         & součin (t-norma)          & definováno & $\max(x+y-1,0)$ \\
      $x \rightarrow y$     & residuum                  & definováno & $\min(1-x+y,1)$ \\
      $ \neg x$             & negace                    & $x \rightarrow 0$ & $1-x$ \\
      $x \leftrightarrow y$ & ekvivalence (biresiduum)  & $(x \rightarrow y) \wedge (y \rightarrow x)$ & $1-|x-y|$ \\
      $x \circ y$           & vzdálenost                & $\neg (x \leftrightarrow y)$ & $|x-y|$ \\
      $x \oplus y$          & součet                    & $\neg (\neg x \otimes \neg y)$ & $\min(x+y,1)$ \\
      $x \ominus y$         & rozdíl                    & $x \otimes \neg y$ & $\max(x-y,0)$ \\
      $x \odot n$           & celočíselný násobek       & $\underbrace{x \oplus x \oplus ... \oplus x}_n, n \in \mathbb{N}$ & $\min(nx,1)$ \\
      $x^n$                 & celočíselná mocnina       & $\underbrace{x \otimes x \otimes ... \otimes x}_n, n \in \mathbb{N} $ & $\max(nx-n+1,0)$ \\
      \bottomrule 
    \end{tabular}
    \caption{Odvozené operátory v \LAsq}
    \end{center}
    \end{table}
    
    
    
\section{Citlivost}

% Celočíselná aproximace -- zmínka o různých typech (int, char)
% jednoduché úvahy o přesnosti, celočíselně kvůli rychlosti
% ztráty odmocňováním -- odmocnina se dá propasovat nahoru -- ztratí se nejméně (pouze 1 LSB)



\section{Reprezentace obrazových dat}

% MASKA:
% co to je, váhy (opakovaný výběr), tvoří se z ní seznam jako základní prvek k zpracování
% okraje
% váhy mohou jít proti robustnosti, hrání s váhami

\section{Filtry}\label{Filtry}
    % něco se dělá nad jedním nebo někalika seznamy a je realizovatelné pomocí FLF, je filtr
    \subsection{Typologie filtrů}       % (implementační hledisko)
        \subsubsection{Morfologické}
        \subsubsection{Statistické}
    \subsection{Meze citlivosti filtrů}



% Kvalita obrázku:

viziualizovat rozdíly mezi filtry, jako "standardy" brát výsledky předchozích prací 