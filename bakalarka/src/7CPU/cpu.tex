% -*-coding: utf-8 -*-

        % Celočíselná aritmetika
            % Přesnost
        % Práce s 3D obrazovými daty
            % Načítání, ukládání (lehce k .hdr, .img)
            % Uchování dat v paměti (řešení okrajů...)
        % Masky, strukturní elementy
        % Implementace filtrů
            % Quickselect
%=========================================================================

V této kapitole popíšeme implementaci filtrů a nejdůležitějších podpůrných struktur na CPU v \Cpp. Před samotným popisem kódu ještě prodiskutujeme téma přesnosti výpočtů.

    \section{Celočíselná aritmetika}

    Při implementaci jsme se omezili pouze na dva celočíselné typy {\tt unsigned char} (0-255) a {\tt unsigned short} (0-65535). Jednak proto, že vstupní data jsou často právě v rozsahu 0-255 a zadruhé, lidské oko stejně není schopné objektivně rozeznat spojitější škálu -- zvláště u jednokanálových (odstíny šedi) obrazů. Neceločíselné typy jsme úplně vynechali ze stejného důvodu, počítání s nimi je navíc podstatně pomalejší. Výběr konkrétního datového typu lze v kódu jednoduše provést pomocí parametru šablony.

        \subsection{Přesnost}

        Jedinou operací, která vnáší do výpočtů chybu, je celočíselné dělení\footnote{ostatní operace ovšem přenášejí chyby již vzniklé} -- násobení, sčítání a odčítání jsou přesné a jiné operace nepoužíváme. Díky požadavkům na teorii (uzavřenost vůči operacím) nikdy nebude docházet k chybám z ořezání výsledku do rozsahu datového typu, nebo jeho přetečení, a můžeme tak analyzovat pouze zaokrouhlovací chyby:

        Hrubý odhad chyby dělení nám dává následující nerovnost:
        \beq
        \frac{m}{n} \le \Big\lfloor \frac{m}{n} \Big\rfloor +1 \qquad m \in \Nn_0, \, n \in \Nn
        \eeq
        Abychom chybu zbytečně nezveličovali, budeme dále uvažovat $\delta \in \RR^+$ a přesnější odhad:
        \beq
        \frac{m}{n} = \Big\lfloor \frac{m}{n} \Big\rfloor + \Big\{ \frac{m}{n} \Big\} = \Big\lfloor \frac{m}{n} \Big\rfloor + \delta
        \eeq
        Jednoduchou úvahou dospějeme k závěru, že největší možná desetinná část při dělení $n$ je $\frac{n-1}{n}$, a tedy:
        \beq
        \delta \leq \frac{n-1}{n}
        \eeq

        Chyba dělění je navíc vždy kladná, a násobení a sčítání jí kladnou nechá. Pokud nepoužijeme ve výpočtech rozdíl, který znaménko chyby samozřejmě nezachová, můžeme toho v odhadech využít. Aritmetiku chyby (obecně se znaménkem) shrnuje tabulka~\ref{tabulka max chyby}. Všechny operace v \LAsq můžeme redukovat na operace uvedené v tabulce (viz výše).

        \begin{table}[h]\label{tabulka max chyby}
    \begin{center}
    \begin{tabular}{lll}
      \toprule
      Operace & Popis & Chyba\\
      \midrule
      $\frac{m\pm\delta_1}{n} \; n \neq 0 $ & Celočíselné dělení konstantou & $\pm \Big( \frac{\delta_1}{n} + \frac{n-1}{n}\Big)$ \\
      $(m\pm\delta_1)+(n\pm\delta_2) $      & Sčítání                       & $\pm \,(\delta_1+\delta_2)$ \\
      $(m\pm\delta_1)-(n\pm\delta_2) $      & Odčítání                      & $\pm \,(\delta_1+\delta_2)$ \\
      $n(m\pm\delta_1) $                    & Násobení konstantou           & $\pm \,n\delta_1$ \\
      $\max(m\pm\delta_1,n\pm\delta_2) $    & Maximum                       & $\pm \max(\delta_1,\delta_2)$ \\
      $\min(m\pm\delta_1,n\pm\delta_2) $    & Minimum                       & $\pm \max(\delta_1,\delta_2)$ \\
      \bottomrule
    \end{tabular}
    \caption{Chyba celočíselných aritmetických operací, $m,n \in \Nn_0, \,\delta_1,\delta_2 \in \RR^+$}
    \end{center}
        \end{table}

        Jako příklad uveďme chybu WBES: na vstupu je použit Walshův list (dělení dvěmi, chyba $0.5$), poté sčítání a dělení čtyřmi. Nepoužíváme rozdíl, takže celočíselný výsledek bude menší nebo rovnen skutečnému:
        \beq
        \delta = +\Big( \frac{0.5+0.5+0.5+0.5}{4} + \frac{4-1}{4} \Big) = +1.25
        \eeq
        Pokud Walshův list uložíme do většího datového typu a průměrování provedeme až na konci (jedinné dělení osmi), můžeme chybu stlačit až na $+0.875$. Dělení (pouze to vnáší chybu) se pomocí rozšíření datového typu obecně snažíme přesouvat až nakonec výpočtu, abychom počítali co nejdéle přesně a minimalizovali tak chybu. Závěrem dodejme, že chyby v příkladu jsou absolutní -- při použití {\tt unsigned char} bude relativní chyba $\frac{+1.25}{256}$, při použití {\tt unsigned short} bude ještě 256krát menší. \notea{ chyba je malá, opakovaným filtrování by došlo k degradaci dat dříve díky rozmazání filtrem, než chybou -- vyzkoušet v příkladě (př 100x WBES char, neoptimalizovaný VS 100 WBES short optimalizovaný)}

    \section{Práce s 3D obrazovými daty}

    O uchovávání, načítání a ukládání 3D dat se v kódu stará šablonová třída \Imageman, geometrii obrazu má na starost abstaktní třída \Imageinfo. Celkový přehled struktury kódu pracujícíh s 3D obrazem lze najít s podrobnějším popisem v příloze~\ref{struktura kódu}. Nyní nám stačí vědět, že 3D obraz je opatřen nulovými okraji (viz sekce~\ref{lokální zprac}) a v paměti uložen jako lineární pole.

    \section{Maska, strukturní element}

    Správu strukturních elementů má na starosti třída {\tt SEManager}, samotné strukturní elementy jsou uloženy ve struktuře {\tt structEl}, kterou pro přehlednost uvádíme zde:

    \begin{Verbatim}[commandchars = \\\{\}]
\bl{typedef struct} _structEl \{
    \bl{string} name;       // název pro pozdější vyhledání
    \bl{int} *wList;        // Lw z teorie, pole o délce = capacity
    \bl{unsigned} capacity; // kapacita masky
    \bl{unsigned} *mask;    // váhy okolních voxelů, neparsovaný vstup, viz dále
\} structEl;
    \end{Verbatim}

    \subsection{Formát}

    Všechny strutury se snažíme optimalizovat na rychlost. Při filtraci budeme chtít rychle a elegantně přistupovat k okolí zpracovávaného voxelu, nejlépe takto (za předpokladu, že {\tt src} ukazuje na zpracovávaný obrázek a {\tt SE} je strukturní element):

    \begin{Verbatim}
iTyVoxelOkoli = src[indexZpracVoxelu + SE.wList[i]]; // 0<= i<SE.capacity
    \end{Verbatim}

    Atribut {\tt wList} bude tedy obsahovat rozdíly indexů centrálního voxelu a okolních voxelů. Masku chceme zadávat v co možná nejjednodušším formátu, jako vhodně uspořádané jednorozměrné pole (přílkad maska $3\times 3\times3$ -- poloměr $R = 1$):

    \begin{Verbatim}[commandchars = \\\{\}]
\bl{unsigned} maska[] = \{
    0,1,0,  1,1,1,  0,1,0,
    1,1,1,  1,4,1,  1,1,1,
    0,1,0,  1,1,1,  0,1,0
\};
    \end{Verbatim}

    Na masku se při zápisu díváme shora a zapisujeme jednotlivé vrstvy, jak je vidíme: nejlevější blok je horní vrstva, nejpravější spodní. Výhodou tohoto formátu je, že pokud bychom chtěli načítat masku ze souboru (neimplementováno), můžeme ji zapsat zcela stejně, protože při klasickém načítání po řádcích se formát nepoškodí. Maska s poloměrem $R = 2$ bude mít analogicky $5$ bloků $5\times 5$.

    Popis načítání masek a další práce s nimi je v příloze~\ref{struktura kódu}.

    \section{Implementace filtrů}

    Filtry jsou obsaženy v abstraktní šablonové třídě {\tt Filter}, což je vlastně pouze kontejner na funkce s několika podpůrnými statickými proměnnými. Před voláním jakékoliv funkce je nutno tyto inicializovat pomocí {\tt Init(sem)}, které se předá ukazatel na {\tt SEManager} (musí již být konkretizovaný). Inicializace dále dopočítá geometrické konstanty použité při procházení datového pole na CPU. Popis formátu filtrů je opět v příloze~\ref{struktura kódu}.

    \subsection{Implementace a optimalizace na CPU}

        Nyní probereme postupně implementační a optimalizační záležitosti vztahující se na veškerý kód na CPU a posléze implementaci jednotlivých filtrů.

        \paragraph{Obecně} se snažíme optimalizovat na rychlost. Vyhýbáme se častým alokacím a dealokacím proměnných v cyklech a opakovaně volaných funkcích, k čemuž \Cpp možností definovat proměnnou kdekoliv celkem svádí -- překladač sice dost těchto případů optimalizuje sám, ale není od věci mu naznačit, že pamětí může plýtvat. \note{prakticky vyzkoušet?}. Taktéž se při větvení programu vyplatí tvořit stromy podmínek spíše do hloubky, než do šířky -- snížíme průměrný počet podmínek, kterými je třeba projít. Toto se samozřejmě týká časově náročných částí kódu, v těch časově nepodstatných je naopak lepší upřednostnit přehlednost. \note{není to už moc obecné?}

        \paragraph{Sekvenční zpracování 3D dat} zajišťuje pro všechny filtry parametrické makro {\tt BEGINFOR3D(pos)} a {\tt ENDFOR3D}, díky kterému projde parametr {\tt pos} pouze ty správné indexy z pole dat. Z celého pole 3D dat totiž musíme zpracovat pouze obraz, nikoliv okraje. Makro obsahuje tři for-cykly a počítá index {\tt pos}, přičemž minimalizuje počet operací k tomu potřebných. Použití ({\tt src} je opět zdroj):

        \begin{Verbatim}[commandchars = \\\{\}]
\bl{unsigned long} idx;
BEGINFOR3D(idx)
    // tělo filtru
    // zpracovávaný voxel je src[idx]
ENDFOR3D
        \end{Verbatim}

        \paragraph{Morfologické filtry} eroze a dilatace jsou implementovány zcela v souladu s definicí v \ref{Filtry}. Pouze detektor hran je lehce optimalizován -- operaci $\DD(\PP) \ominus \EE(\PP)$ můžeme samozřejmě přesunout na úroveň voxelů:

        \begin{Verbatim}[commandchars = \\\{\}]
\bl{unsigned long} idx;
\bl{unsigned} i;
\bl{imDataType} min,max;
BEGINFOR3D(idx)
    min = max = src[idx + SE.wList[0]];
    \bl{for}(i=1; i<SE.capacity; i++)\{
        \bl{if}(min > src[idx + SE.wList[i]]) min = src[idx + SE.wList[i]];
        \bl{if}(max < src[idx + SE.wList[i]]) max = src[idx + SE.wList[i]];
    \}
    dst[idx] = max - min;
ENDFOR3D
        \end{Verbatim}

        Zmíněné tři filtry jsou implementačně triviální, zajímavé ale bude jejich urychlení na GPU, kde se vzhledem k jednoduchosti operací dostaneme na hranici hardwarových možností.

        \paragraph{Statisticky motivované filtry} jsou implementačně nejzajímavější, neboť je při nich třeba z pole hodnot vybrat obecně několikrát \kk-tý prvek. Implementovány jsou čtyři základní filtry: {\tt Median}, {\tt WMedian}, {\tt BES} a {\tt WBES}. Jejich jádrem na CPU jsou dvě funkce {\tt MedianFindOpt} a {\tt UniBESFind}, které operují buď nad polem vytvořeným z obrazových dat podle {\tt wList}, nebo nad příslušným Walshovým seznamem\footnote{seznam je zde pouze součást názvu, samozřejmě je implementován jako pole} vytvořeným z tohoto pole.

        Funkce {\tt MedianFindOpt(base,length)} částečně setřídí pole {\tt base} o délce {\tt length} tak, že se na indexu {\tt length/2} bude nacházet medián (případně na indexech {\tt (length/2)} a {\tt (length/2)-1} prvky potřebné ke konstrukci mediánu pro {\tt length} sudé).

        \begin{Verbatim}[commandchars = \\\{\}]
\bl{bool} MedianFindOpt(\bl{imDataType} *base, \bl{unsigned} initBaseLength)\{
    \bl{static int} first, last;
    \bl{static unsigned} length = 0, rising, falling, pivot, itemsToFind, upper, lower;
    \bl{static imDataType} swap;
    \bl{if}(initBaseLength != 0)\{      //inicializace
        length = initBaseLength;
        upper = length/2;
        lower = length/2-1+length\%2;  //lichá délka bude stejný jako upper
        \bl{return true};
    \}
    itemsToFind = 2-initBaseLength\%2; //2 nebo 1 prvek
    first = 0;	
    last = length-1;

    \bl{while}(1)\{                 //vnější smyčka
        \bl{if}(!progress) \bl{return true};
        \bl{if}(last-first < EFFICIENT_MAX)\{
            \bl{if}(last-upper < lower - first)
                InsertSortMax(base,first,last,last-lower+1);	
            \bl{else}
                InsertSortMin(base,first,last,upper-first+1);	
            \bl{return true};
        \}
        pivot = first;
        rising = first+1;
        falling = last;
        \bl{if}(base[rising] > base[falling]) SWAP(rising,falling);
        \bl{if}(base[rising] > base[pivot]) SWAP(rising,pivot);
        \bl{if}(base[falling] < base[pivot]) SWAP(falling,pivot);

        \bl{while}(1)\{			     //vnitřní smyčka				
            \bl{do} falling--; \bl{while}(base[falling] > base[pivot]);
            \bl{do} rising++; \bl{while}(base[rising] < base[pivot]);
            \bl{if}(rising > falling)\{	
                SWAP(pivot,falling);
                \bl{break};
            \}\bl{else}\{					
                SWAP(rising,falling);
            \}
        \}
        \bl{if}(falling > upper) last = falling-1;	
        \bl{else if}(falling < lower) first = falling+1;
        \bl{else}\{
            progress--;
            \bl{if}(falling == lower) first = falling+1;
            \bl{else} last = falling-1;
        \}
    \}
\}
        \end{Verbatim}

        Funkce je volána při zpracování každého voxelu, proto jsou proměnné alokovány staticky, tudíž se nedealokují po opuštění funkce a šetří čas při správě paměti. K nalezení požadovaných prvků je použit algoritmus \emph{Quickselect} založený a \emph{Quicksort}u, který je i s optimalizacemi popsán v \cite{Numerical Recipes}. V základní verzi algoritmus pracuje následovně:
        \begin{enumerate}
          \item V poli vybereme pivota (obyčejně nejlevější prvek) {\tt p}
          \item Zbytek pole procházíme ukazatelem zleva, dokud nenajdeme prvek $\ge$ {\tt p}
          \item Stejně procházíme zprava, dokud nenajdeme prvek $\le$ {\tt p}
          \item Tyto dva prvky prohodíme
          \item Pokračujeme dokud se ukazatele nepřekříží, na toto místo vložíme {\tt p}
          \item Stejný postup aplikujeme na levé, nebo pravé podpole (pivot je již na svém místě), podle toho, jesti pivot skončil pod, nebo nad mediánem (obecně \kk-tým prvkem)
        \end{enumerate}

        Časově nejnáročnější je vnitřní smyčka algoritmu, tu můžeme urychlit odebráním kontrol, zda jsme s ukazatelem nevyjeli mimo pole, což se může stát, pokud by byl pivot nejmenším, nebo největším prvkem v prohledávaném úseku. To vyřešíme tak, že pivota zvolíme jako medián tří prvků, z nichž největší umístíme nakonec prohledávaného úseku a nejmenší na začátek -- tím pádem nám indexy nikdy z pole neutečou. Cenou za toto urychlení je, že musíme měnit i prvky stejné jako pivot, protože nemůžeme zaručit, že medián budeme dělat z tří \emph{různě velkých} prvků (indexy by pak zase mohly utéct). I tak je však však zrychlení vůči verzi s kontrolou mezí více než dvojnásobné.

        Poslední optimalizací je použití alternativního třídícího algoritmu (v našem případě Insertion sort), když je tříděný úsek příliš krátký, protože tam je Quickselect neefektivní. Vzhledem k tomu, že potřebujeme správně umístěné pouze dva prvky, setřídíme Insertion sortem pouze nekratší možnou část pole (buď zprava, nebo zleva).

        Funkce {\tt UniBESFind(base,length)} je prakticky stejná jako {\tt MedianFindOpt(base,length)}, jen ještě správně umístí první a třetí kvartil: Nejprve najde medián (nebo potřebné dva prvky) a přitom si zapamatuje pivoty (už jsou správně umístěné), které jsou nejblíže příslušnému kvartilu. Ten pak hledá už jen mezi nimi (nebo mezi pivotem a koncem pole). K tomu používá funkci {\tt FindKth(base,first,last,k)}, která opět pracuje nad polem {\tt base} a hledá pomocí quickselectu \kk-tý prvek mezi indexy {\tt first} a {\tt last}. Funguje podobně jako {\tt MedianFindOpt}, používá v poslední fázi insertion sort, jen je optimalizovaná pro nalezení pouze jednoho prvku.
























