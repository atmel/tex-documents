% -*-coding: utf-8 -*-

% cíl, trend

% aplikace zpracovává 3D obraz

Práce je zaměřená na urychlení známých obrazových filtrů na GPU. Přesouvání výpočtů na GPU má dnes už ve vědecké komunitě své pevné místo a k řadě problémů jsou již k dispozici knihovny GPU využívající (třídění polí, konvoluce v obraze). Náš problém se však menšími rozměry vstupních dat -- je třeba třídit pole pouze o stovkách prvků -- tomuto mainstreamu poněkud vymyká. Algoritmy, které urychlí fitraci na dobu mžiku (ms až desítky ms) mohou být výhledově použity v aplikacích, které potřebují výsledky filtrů v reálném čase -- např. program umožňující laikovi sestavit si potřebný filtr pomocí jednoduchého skládání základních filtrů a manipulací s jejich parametry, kde potřebuje po každé změně okamžitou zpětnou vazbu. 

Jako solidní teoretický základ filtrů byla použita fuzzy logika, konkrétně \L ukasiewiczova algebra s odmocninou \cite{MajerovaPhD}, která je přímo aplikovatelná na data tak, jak jsou v počítači reprezentována a je vhodná pro konstrukci sítí filtrů, při jíž se může urychlení na GPU taktéž bohatě uplatnit.
Popis je zaměřen na typové filtry ze tříd morfologický a statisticky motivovaných filtrů, které jsou posléze i implementovány.

V práci nebudeme porovnávat urychlení konkrétních algoritmů mezi CPU a GPU, ale uryhclení celých filtrů. Při implementaci jsou tak na CPU a GPU použity pro ten samý filtr vždy rozdílné algoritmy, z nichž každý je optimalizován na rychlost a konkrétní platformu -- díky hardwarovým omezením na GPU klademe na kód jiné požadavky.

Implementace na CPU je provedena v jazyce \Cpp ~za použítí nejlepších dostupných algoritmů \cite{Numerical Recipes}, na GPU je použito prostředí NVIDIA CUDA C a použité algoritmy a optimaliazce jsou až na výjimky původní.