% -*-coding: utf-8 -*-

\section{Specifikace požadavků na model}

Požadavky na model můžeme rozdělit na koncepční a implementační. \note{model je hodně spjat s implementací, jak ho naformulovat jako teoretický...}

\subsubsection{Koncepční požadavky}

\note{program je vlastně hromada univerzálních dílčích komponent a to, co musíme dodat, abychom dostali požadovaný algoritmus je jen myšlenkové "lepidlo" -- prostřední komponenty určující, jak se budou základní komponenty chovat -- př. merge s elitismem (způsob nahrazení staré populace novou) určuje odkud kam a co se bude třídit a kam se vytříděné bude přesouvat}

\note{Jak tohle vhodně formulovat.}


\begin{itemize}
  \item 
\end{itemize}

koncepční:
Rozklad algoritmů na znovu použitelné komponenty
jednoduchá tvorba nových komponent -- odstínit uživatele od implementačních detailů
možnost zapouzdřit jednotlivé myšlenky do komponent
sériové i paralelní výpočty

implementační:
možnost sériových výpočtů na CPU, malé zhoršení oproti ideální implementaci (neberoucí ohled na GPU)
runtime konstrukce algoritmu
čitelný výstup (vizualizace v matlabu)

\section{Konstrukce modelu}