% -*-coding: utf-8 -*-

Podařilo se nám vytvořit vysoce funkční model pro paralelizaci optimalizačních algoritmů. V popsaném formalismu založeném na struktuře evolučních algoritmů jsme schopni popsat, rozložit a posléze v modelu implementovat celou škálu optimalizačních algoritmů. Díky důrazu na stavebnicový princip můžeme algoritmus dekomponovat na malé části, jejichž implementace a paralelizace je jednoduchá, nebo je možné při ní použít osvědčené algoritmy. Vzniklé komponenty jsou elegantně malé opakovaně použitelné i v jiných OA.

Při testování zrychlení za použití GPU na třech algoritmech (RS, GO, SA) a na De Jongově účelové funkci č.1 dosáhli očekávaných výsledků zrychlení na úrovni 100. Na problém sudoku bylo dosaženo výsledků překvapivých, kdy maximální zrychlení dosáhlo hodnoty 566krát, tedy nad očekávanou maximální hranicí odpovídající počtu jader na GPU (336). To však může být způsobeno neoptimálním kódem na pro CPU (byť je na GPU použit stejný), nebo projevem větší propustnosti paměti GPU, která je při výpočtu účelové funkce intenzivně využívána.

Navržený model je snadno rozšiřitelný, životaschopný a může sloužit jako základní kámen pro statistické testování kvality různých OA, které vyžaduje extenzivní výpočty. Zvláště výsledky pro De Jongovu funkci jsou slibné, neboť ostatní funkce z De Jongova testovacího souboru jsou podobně výpočetně jednoduché.