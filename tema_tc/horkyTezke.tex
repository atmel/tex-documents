% -*-coding: utf-8 -*-
\documentclass[12pt, a4paper]{report}

% JAZYK, FONTY, LAYOUT
\usepackage[czech]{babel}
\usepackage[utf8]{inputenc}
\usepackage[T1,IL2]{fontenc}
\usepackage[margin=1in]{geometry}

% MATEMATICKÉ DEFINICE
% POUŽÍVEJTE PROSTŘEDÍ theorem, lemma, definition, remark, example, NEBO SI DEFINUJTE NOVÉ
\usepackage{amsmath,amsfonts,amsthm,mathtools,upgreek}
\def\proofname{{\scshape Důkaz.}}

\begin{document}

Ladislav Horký \hfill \today
\section{Těžký příklad}
Dokažte, že cirkulační matice $C$ má vlastní čísla ve tvaru
\[
\lambda = \sum_{k=0}^{n-1} c_k \omega^k,
\]
kde $\omega$ je n-tý primitivní kořen z jedničky.
\[C=
\begin{pmatrix}
  c_0 & c_1 & c_2 & \cdots & c_{n-1} \\
  c_{n-1} & c_0 & c_1 & \cdots & c_{n-2} \\
  c_{n-2} & c_{n-1} & c_0 & \cdots & c_{n-3} \\
  \vdots & \vdots & \vdots & \ddots & \vdots \\
  c_1 & c_2 & c_3 & \cdots & c_0 \\
\end{pmatrix}
\]

\vspace{1cm}
Rozepíšeme, co znamená, že $C$ má vlastní číslo $\lambda$ po složkách,
\[
Cy = \lambda y,
\]
pro m-tý řádek tedy:
\[
\sum_{k=n-m+1}^{n} c_k y_{k-n+m} + \sum_{k=0}^{n-m} c_k y_{k+m} = \lambda y_m
\]
Nyní zkusíme uhodnout $y_m$. Vidíme, že kdybychom to, co je v indexu přesunuli do mocniny, hezky by se krátilo. 
Zkusme tedy $y_m = \alpha^m$ přičemž podmínka na $\alpha$ vypadne z toho, že $y$ má být vlastní vektor:
\[
\sum_{k=0}^{n-m} c_k \alpha^{k+m} + \sum_{k=n-m+1}^{n} c_k \alpha^{k-n+m} = \lambda \alpha^m
\]
\[
\sum_{k=0}^{n-m} c_k \alpha^{k} + \alpha^{-n}\sum_{k=n-m+1}^{n} c_k \alpha^{k} = \lambda
\]
\[
\sum_{k=0}^{n} c_k \alpha^{k} + \underbrace{(1-\alpha^{-n})\sum_{k=n-m+1}^{n} c_k \alpha^{k}}_
{\text{pouze toto závisí na $m$}} = \lambda \quad\forall m.
\]
Poslední člen má být konstantní $\forall m$, z čehož plyne, že závorka před sumou musí být rovna 0. Tedy $\alpha$ můžeme volit jako primitivní kořen z jedničky ($=\omega$). Otázkou je, zda máme dost LN vlastních vektorů pro více vlastních čísel - vidíme však, že výše zmíněné podmínce odpovídají volby $\alpha = \omega^0, \omega, \omega^2, ... \omega^n-1$, tzn. vlastní vektory vypadají:
\[
\begin{pmatrix}
  1 \\ 1 \\ 1 \\ \vdots \\ 1
\end{pmatrix},
\begin{pmatrix}
  1 \\ \omega \\ \omega^2 \\ \vdots \\ \omega^{n-1}
\end{pmatrix},
\begin{pmatrix}
  1 \\ \omega^2 \\ \omega^4 \\ \vdots \\ \omega^{n-1}
\end{pmatrix}
\] 

Další vlastní čísla (nejvýše $n$) získáme dosazením mocnin $\alpha$. Dodatkem budiž, že po dosazení $\alpha$ představuje poslední rovnice vlastně Fourierovu transformaci posloupnosti $(c_i)_{i=0}^{n-1}$.

\end{document} 